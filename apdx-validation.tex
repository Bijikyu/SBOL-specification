\newcounter{sbolCtr}
\newcommand{\printValid}{\validRule{sbol-\arabic{sbolCtr}\addtocounter{sbolCtr}{1}}}
\newcommand{\printModeling}{\modelingRule{sbol-\arabic{sbolCtr}\addtocounter{sbolCtr}{1}}}

\section{Validation Rules}
\label{validation}

\Rtodo{Someone please review.}

This section summarizes all the conditions that must (or in some cases,
at least \emph{should}) be true of an SBOL Version~2 document.  
There are different degrees of rule strictness.  Formally, the differences are expressed in the statement of a rule: either a rule states that a condition \emph{must} be true, or a rule states that it \emph{should} be true.  Rules of the former kind are strict SBOL validation rules---data encoded in SBOL must conform to
all of them in order to be considered valid.  Rules of the latter kind
are consistency rules.  To help highlight these differences, we use the
following three symbols next to the rule numbers:

\begin{description}

\item[\hspace*{6.5pt}\vSymbol\vsp] A \vSymbolName indicates a
  \emph{requirement} for SBOL conformance. If a SBOL document does not follow this rule, it does not conform to the SBOL
  specification.  (Mnemonic intention behind the choice of symbol:
  ``This must be checked.'')

\item[\hspace*{6.5pt}\cSymbol\csp] A \cSymbolName indicates a
  \emph{recommendation} for consistency.  If a SBOL document does not
  follow this rule, it is not considered strictly invalid as far as the
  SBOL specification is concerned; however, it indicates that the model contains a physical or conceptual inconsistency.  (Mnemonic intention behind the choice of symbol:
  ``This is a cause for warning.'')

\item[\hspace*{6.5pt}\mSymbol\msp] A \mSymbolName indicates a strong
  recommendation for best practice.  This rule is not strictly
  a matter of SBOL encoding, but the recommendation comes from logical
  reasoning.  As in the previous case, if a SBOL document does not follow this rule, it is not considered an invalid SBOL encoding.  (Mnemonic intention behind the choice of symbol: ``You're a star if you heed this.'')

\end{description}

The validation rules listed in the following subsections are all stated
or implied in the rest of this specification document.  They are
enumerated here for convenience.  

For \notice convenience and brievity, we use the shorthand
``\token{sbol:x}'' to stand for an attribute or element name \token{x}
in the namespace for the SBOL specification, using
the namespace prefix \token{sbol}.  In reality, the prefix string may be different from the literal ``\token{sbol}'' used here (and indeed, it can be any valid XML namespace prefix that the software
chooses).  We use ``\token{sbol:x}'' because it is shorter than to
write a full explanation everywhere we refer to an attribute or element
in the SBOL specification namespace.

\subsubsection*{General rules about an SBOL document}
\setcounter{sbolCtr}{10101} 

\printValid{An SBOL document must declare the use of the following XML Namespace: \\ \textls[-25]{\uri{http://sbols.org/v2\#}}. (Reference:
  \sec{xml-namespace}.)}

\printValid{An SBOL document must declare the use of the following XML Namespace: \\ \textls[-25]{\uri{http://www.w3.org/1999/02/22-rdf-syntax-ns\#}}. (Reference: \sec{xml-namespace}.)}

\printValid{An SBOL document must declare the use of the following XML Namespace: \\ \textls[-25]{\uri{http://purl.org/dc/terms/}}. (Reference: \sec{xml-namespace}.)}

\printModeling{An SBOL document should declare the use of the following XML Namespace: \\ \textls[-25]{\uri{http://www.w3.org/ns/prov\#}}. (Reference: \sec{xml-namespace}.)}

\subsubsection*{Rules for the \class{Identified} class} 
\setcounter{sbolCtr}{10201}

\printValid{The \sbol{identity} is a required property for all \sbol{Identified} objects and has a data type of URI with a syntax defined by:\\
\uri{http://www.w3.org/1999/02/22-rdf-syntax\#about} (Reference: \sec{sec:Identified})}

\printValid{The \sbol{persistentIdentity} is an optional property for all \sbol{Identified} objects and, if provided, has a data type of URI with a syntax defined by:\\ \uri{http://www.w3.org/1999/02/22-rdf-syntax\#about} (Reference: \sec{sec:Identified})}

\printValid{The \sbol{displayId} is an optional property for all \sbol{Identified} objects and, if provided, has a data type of String that is composed only of alphanumeric or underscore characters and must not begin with a digit. (Reference: \sec{sec:Identified})}

\printValid{The \sbol{version} is an optional property for all \sbol{Identified} objects and, if provided, has a data type of String that is composed only of alphanumeric characters, underscores, hyphens, and periods and must begin with a digit. (Reference: \sec{sec:Identified})}

\printValid{The \sbol{annotations} field is an optional list of for all \sbol{Identified} objects and, if provided, includes references to \sbol{Annotation} objects. (Reference: \sec{sec:Identified})}

\printModeling{The \sbol{identity} of a compliant object must either be equal to the \sbol{persistentIdentity} when no \sbol{version} is specified or equal to "\refObj{persistentIdentity}/\refObj{version}" when a \sbol{version} is provided. (Reference: \sec{sec:compliant})}

\printModeling{The \sbol{persistentIdentity} of a compliant \sbol{Collection} object is required and must end with "/col/\refObj{displayId}". (Reference: \sec{sec:compliant})}

\printModeling{The \sbol{persistentIdentity} of a compliant \sbol{ModuleDefinition} object is required and must end with "/md/\refObj{displayId}". (Reference: \sec{sec:compliant})}

\printModeling{The \sbol{persistentIdentity} of a compliant \sbol{Model} object is required and must end with\\ "/mod/\refObj{displayId}". (Reference: \sec{sec:compliant})}

\printModeling{The \sbol{persistentIdentity} of a compliant \sbol{ComponentDefinition} object is required and must end with "/cd/\refObj{displayId}". (Reference: \sec{sec:compliant})}

\printModeling{The \sbol{persistentIdentity} of a compliant \sbol{Sequence} object is required and must end with "/seq/\refObj{displayId}". (Reference: \sec{sec:compliant})}

\printModeling{The \sbol{persistentIdentity} of a compliant \sbol{GenericTopLevel} object is required and must end with "/gen/\refObj{displayId}". (Reference: \sec{sec:compliant})}

\printModeling{The \sbol{persistentIdentity} of a compliant child object is required must begin with the\\ \sbol{persistentIdentity} of its parent object and be immediately followed by "/\refObj{displayId}". (Reference: \sec{sec:compliant})}

\printModeling{The \sbol{version} of a compliant child object is required to be equal to the \sbol{version} of its parent object. (Reference: \sec{sec:compliant})}

\printModeling{The \sbol{displayId} of a compliant object is required.  (Reference: \sec{sec:compliant})}

\printValid{The \sbol{name} is an optional property for all \sbol{Identified} objects and, if provided, has a data type of String.  (Reference: \sec{sec:Identified})}

\printValid{The \sbol{description} is an optional property for all \sbol{Identified} objects and, if provided, has a data type of String.  (Reference: \sec{sec:Identified})}

\subsubsection*{Rules for the \class{TopLevel} class} 
\setcounter{sbolCtr}{10401}

\printValid{A \sbol{TopLevel} object inherits all properties of a \sbol{Identified} object. (Reference: \sec{sec:TopLevel})}

\subsubsection*{Rules for the \class{Sequence} class} 
\setcounter{sbolCtr}{10501}

\printValid{A \sbol{Sequence} object inherits all properties of a \sbol{TopLevel} object. (Reference: \sec{sec:Sequence})}

\printValid{The \sbol{elements} is a required property for all \sbol{Sequence} objects and has a data type of String.  Its format must be consistent with its specified \sbol{encoding}. (Reference: \sec{sec:Sequence})}

\printValid{The \sbol{encoding} is a required property for all \sbol{Sequence} objects and has a data type of URI. (Reference: \sec{sec:Sequence})}

\printModeling{A DNA \sbol{Sequence} should use the IUPAC DNA encoding:\\ \url{http://www.chem.qmul.ac.uk/iubmb/misc/naseq.html}. (Reference: \sec{sec:Sequence})}

\printModeling{A RNA \sbol{Sequence} should use the IUPAC RNA encoding:\\ \url{http://www.chem.qmul.ac.uk/iubmb/misc/naseq.html}. (Reference: \sec{sec:Sequence})}

\printModeling{A protein \sbol{Sequence} should use the IUPAC Protein encoding:\\ \url{http://www.chem.qmul.ac.uk/iupac/AminoAcid/}. (Reference: \sec{sec:Sequence})}

\printModeling{A small molecule \sbol{Sequence} should use the SMILES encoding:\\ \url{http://www.opensmiles.org/opensmiles.html}. (Reference: \sec{sec:Sequence})}

\subsubsection*{Rules for the \class{ComponentDefinition} class} 
\setcounter{sbolCtr}{10601}

\printValid{A \sbol{ComponentDefinition} object inherits all properties of a \sbol{TopLevel} object. (Reference: \sec{sec:ComponentDefinition})}

\printValid{The \sbol{types} property is a set of URIs, and it is required to include at least entry. (Reference: \sec{sec:ComponentDefinition})}

\printValid{The \sbol{roles} property is an optional set of URIs.  (Reference: \sec{sec:ComponentDefinition})}

\printValid{The \sbol{components} property is an optional set of \sbol{Component} objects.  (Reference: \sec{sec:ComponentDefinition})}

\printValid{The \sbol{sequenceConstraints} property is an optional set of \sbol{SequenceConstraint} objects.  (Reference: \sec{sec:ComponentDefinition})}

\printValid{The \sbol{sequenceAnnotations} property is an optional set of \sbol{SequenceAnnotation} objects.  (Reference: \sec{sec:ComponentDefinition})}

\printValid{The \sbol{sequence} property is an optional URI reference to a \sbol{Sequence} object.  (Reference: \sec{sec:ComponentDefinition})}

\printModeling{The \sbol{sequence} property URI, if specified, should reference a \sbol{Sequence} object.  (Reference: \sec{sec:ComponentDefinition})}

\printValid{A DNA \sbol{ComponentDefinition} must include the URI\\ 
\url{http://www.biopax.org/release/biopax-level3.owl\#DnaRegion} 
in the set of \sbol{types}.  (Reference: \sec{sec:ComponentDefinition})}

\printValid{A RNA \sbol{ComponentDefinition} must include the URI\\ 
\url{http://www.biopax.org/release/biopax-level3.owl\#RnaRegion} in the set of \sbol{types}.  (Reference: \sec{sec:ComponentDefinition})}

\printValid{A protein \sbol{ComponentDefinition} must include the URI\\ 
\url{http://www.biopax.org/release/biopax-level3.owl\#Protein} in the set of \sbol{types}.  (Reference: \sec{sec:ComponentDefinition})}

\printValid{A small molecule \sbol{ComponentDefinition} must include the URI\\ 
\url{http://www.biopax.org/release/biopax-level3.owl\#SmallMolecule} in the set of \sbol{types}.  (Reference: \sec{sec:ComponentDefinition})}

\printModeling{A least one type in the set of \sbol{types} should be a URI from the BioPAX or ChEBI ontologies. (Reference: \sec{sec:ComponentDefinition})}

\subsubsection*{Rules for the \class{ComponentInstance} class} 
\setcounter{sbolCtr}{10701}

\printValid{A \sbol{ComponentInstance} object inherits all properties of a \sbol{Identified} object. (Reference: \sec{sec:ComponentInstance})}

\printValid{The \sbol{access} property is a required URI which must be one of the following:
\url{http://sbols.org/v2\#public} or \url{http://sbols.org/v2\#private}.
(Reference: \sec{sec:ComponentInstance})}

\printValid{The \sbol{definition} property is a required URI reference to a \sbol{ComponentDefinition} object.  (Reference: \sec{sec:ComponentInstance})}

\printModeling{The \sbol{definition} property URI should reference a \sbol{ComponentDefinition} object. (Reference: \sec{sec:ComponentInstance})}

\printValid{The \sbol{mapsTos} property is an optional set of \sbol{MapsTo} objects.  (Reference: \sec{sec:ComponentInstance})}

\printValid{A \sbol{Component} object inherits all properties of a \sbol{ComponentInstance} object. (Reference: \sec{sec:ComponentInstance})}

\printValid{A \sbol{FunctionalComponent} object inherits all properties of a \sbol{ComponentInstance} object. (Reference: \sec{sec:ComponentInstance})}

\printValid{The \sbol{direction} property is a required URI for \sbol{FunctionalComponent} objects which must be one of the following:
\url{http://sbols.org/v2\#inout}, \url{http://sbols.org/v2\#in}, \url{http://sbols.org/v2\#out}, or \url{http://sbols.org/v2\#none}.
(Reference: \sec{sec:FunctionalComponent})}

\subsubsection*{Rules for the \class{SequenceAnnotation} class} 
\setcounter{sbolCtr}{10801}

\printValid{A \sbol{SequenceAnnotation} object inherits all properties of a \sbol{Identified} object. (Reference: \sec{sec:SequenceAnnotation})}

\printValid{The \sbol{locations} property is a set of \sbol{Location} objects, and it must include at least one \sbol{Location}. (Reference: \sec{sec:SequenceAnnotation})}

\printValid{The \sbol{component} property is an optional URI that MUST reference a \sbol{Component} that is specified within the same \sbol{ComponentDefinition}. (Reference: \sec{sec:SequenceAnnotation})}

\subsubsection*{Rules for the \class{Location} class} 
\setcounter{sbolCtr}{10901}

\printValid{A \sbol{Location} object inherits all properties of an \sbol{Identified} object. (Reference: \sec{sec:Location})}

\printValid{The \sbol{orientation} property is an optional URI which, if specified, must be one of the following: \url{http://sbols.org/v2\#inline}, 
\url{http://sbols.org/v2\#reverseComplement}.
(Reference: \sec{sec:GenericLocation})}

\printValid{A \sbol{GenericLocation} object inherits all properties of a \sbol{Location} object. (Reference: \sec{sec:GenericLocation})}

\printValid{A \sbol{Range} object inherits all properties of a \sbol{Location} object. (Reference: \sec{sec:Range})}

\printValid{The \sbol{start} property is required for all \sbol{Range} objects, and it must be a positive integer.  (Reference: \sec{sec:Range})}

\printValid{The \sbol{end} property is required for all \sbol{Range} objects, and it must be a positive integer.  (Reference: \sec{sec:Range})}

\printValid{A \sbol{Cut} object inherits all properties of a \sbol{Location} object. (Reference: \sec{sec:Cut})}

\printValid{The \sbol{at} property is required for all \sbol{Cut} objects, and it must be a non-negative integer.  (Reference: \sec{sec:Cut})}

% \Ctodo{MultiRange must have 2 ranges}

\subsubsection*{Rules for the \class{SequenceConstraint} class} 
\setcounter{sbolCtr}{11001}

\printValid{A \sbol{SequenceConstraint} object inherits all properties of an \sbol{Identified} object. (Reference: \sec{sec:SequenceConstraint})}

\printValid{The \sbol{restriction} property is a required URI which must be one of the following:
\url{http://sbols.org/v2\#precedes},
\url{http://sbols.org/v2\#sameOrientationAs}, or
\url{http://sbols.org/v2\#oppositeOrientationAs}.
(Reference: \sec{sec:SequenceConstraint})}

\printValid{The \sbol{subject} property is an required URI that MUST reference a \sbol{Component} that is specified within the same \sbol{ComponentDefinition}. (Reference: \sec{sec:SequenceConstraint})}

\printValid{The \sbol{object} property is a required URI that MUST reference a \sbol{Component} that is specified within the same \sbol{ComponentDefinition}. (Reference: \sec{sec:SequenceConstraint})}

\subsubsection*{Rules for the \class{Model} class} 
\setcounter{sbolCtr}{11101}

\printValid{A \sbol{Model} object inherits all properties of a \sbol{TopLevel} object. (Reference: \sec{sec:Model})}

\printValid{The \sbol{source} property is a required URI that MUST specify the location of the model source file.}

\printValid{The \sbol{language} property is a required URI that specifies the language in which the model is encoded.}

\printModeling{The \sbol{language} property should be a URI from the EMBRACE Data and Methods (EDAM) ontology.}

\printValid{The \sbol{framework} property is a required URI that specifies the modeling framework.}

\printModeling{The \sbol{framework} property should be a URI from the  modeling framework branch of the Systems Biology Ontology (SBO).}

\subsubsection*{Rules for the \class{ModuleDefinition} class} 
\setcounter{sbolCtr}{11201}

\printValid{A \sbol{ModuleDefinition} object inherits all properties of a \sbol{TopLevel} object. (Reference: \sec{sec:ModuleDefinition})}

\printValid{The \sbol{roles} property is an optional set of URIs.  (Reference: \sec{sec:ModuleDefinition})}

\printValid{The \sbol{modules} property is an optional set of \sbol{Module} objects.  (Reference: \sec{sec:ModuleDefinition})}

\printValid{The \sbol{interactions} property is an optional set of \sbol{Interaction} objects.  (Reference: \sec{sec:ModuleDefinition})}

\printValid{The \sbol{functionalComponents} property is an optional set of \sbol{FunctionalComponent} objects.  (Reference: \sec{sec:ModuleDefinition})}

\printValid{The \sbol{models} property is an optional set of URIs. that reference \sbol{Model} objects.  (Reference: \sec{sec:ModuleDefinition})}

\printModeling{Each URI in the set of \sbol{models} should reference a \sbol{Model} object.  (Reference: \sec{sec:ModuleDefinition})}

\subsubsection*{Rules for the \class{Module} class} 
\setcounter{sbolCtr}{11301}

\printValid{A \sbol{Module} object inherits all properties of a \sbol{Identified} object. (Reference: \sec{sec:Module})}

\printValid{The \sbol{definition} property is a required URI reference to a \sbol{ModuleDefinition} object.  (Reference: \sec{sec:Module})}

\printValid{The \sbol{mapsTos} property is an optional set of \sbol{MapsTo} objects.  (Reference: \sec{sec:Module})}

\subsubsection*{Rules for the \class{MapsTo} class} 
\setcounter{sbolCtr}{11401}

\printValid{A \sbol{MapsTo} object inherits all properties of an \sbol{Identified} object. (Reference: \sec{sec:MapsTo})}

\printValid{The \sbol{refinement} property is a required URI for \sbol{MapsTo} objects which must be one of the following:
\url{http://sbols.org/v2\#useremote},
\url{http://sbols.org/v2\#uselocal},
\url{http://sbols.org/v2\#verifyIdentical}, and
\url{http://sbols.org/v2\#merge}.
(Reference: \sec{sec:MapsTo})}

\printValid{The \sbol{local} property is a required URI that MUST reference a \sbol{FunctionalComponent} that is specified within the same \sbol{ModuleDefinition}. (Reference: \sec{sec:MapsTo})}

\printValid{The \sbol{FunctionalComponent} referenced by the \sbol{local} property must have \sbol{public} access type.}

\printValid{The \sbol{remote} property is a required URI that MUST reference a \sbol{FunctionalComponent} that is specified within the \sbol{ModuleDefinition} referred to by the \sbol{definition} property of the \sbol{Module} that contains this \sbol{MapsTo} object. (Reference: \sec{sec:MapsTo})}

\printValid{The \sbol{FunctionalComponent} referenced by the \sbol{remote} property must have \sbol{public} access type.}

\subsubsection*{Rules for the \class{Interaction} class} 
\setcounter{sbolCtr}{11501}

\printValid{A \sbol{Interaction} object inherits all properties of a \sbol{Identified} object. (Reference: \sec{sec:Interaction})}

\printValid{The \sbol{types} property is a set of URIs, and it is required to include at least entry. (Reference: \sec{sec:Interaction})}

\printModeling{A least one type in the set of \sbol{types} should be a URI from the occurring entity relationship branch of the Systems Biology Ontology (SBO). (Reference: \sec{sec:Interaction})}

\printValid{The \sbol{participations} property is an optional set of \sbol{Participation} objects.  (Reference: \sec{sec:Interaction})}

\subsubsection*{Rules for the \class{Participation} class} 
\setcounter{sbolCtr}{11601}

\printValid{A \sbol{Participation} object inherits all properties of an \sbol{Identified} object. (Reference: \sec{sec:Participation})}

\printValid{The \sbol{participant} property is a required URI that MUST reference a \sbol{FunctionalComponent} that is specified within the same \sbol{ModuleDefinition}. (Reference: \sec{sec:Participation})}

\printValid{The \sbol{roles} property is an optional set of URIs.  (Reference: \sec{sec:Participation})}

\printModeling{A least one role in the set of \sbol{roles} should be a URI from the participant role branch of the Systems Biology Ontology (SBO). (Reference: \sec{sec:Participation})}

\subsubsection*{Rules for the \class{Collection} class} 
\setcounter{sbolCtr}{11701}

\printValid{A \sbol{Collection} object inherits all properties of a \sbol{TopLevel} object. (Reference: \sec{sec:Collection})}

\printValid{The \sbol{members} property is an optional set of URIs. that reference \sbol{TopLevel} objects.  (Reference: \sec{sec:Collection})}

\printModeling{Each URI in the set of \sbol{members} should reference a \sbol{TopLevel} object.  (Reference: \sec{sec:Collection})}

\subsubsection*{Rules for the \class{Annotation} class} 
\setcounter{sbolCtr}{11801}

\Ctodo{Not sure what the rules are for this one}

\subsubsection*{Rules for the \class{GenericTopLevel} class} 
\setcounter{sbolCtr}{11701}

\printValid{A \sbol{GenericTopLevel} object inherits all properties of a \sbol{TopLevel} object. (Reference: \sec{sec:GenericTopLevel})}

\printValid{The \sbol{rdfType} property is required, and it has data type QName. (Reference: \sec{sec:GenericTopLevel})}