% -----------------------------------------------------------------------------
\section{A Brief History of SBOL}
% -----------------------------------------------------------------------------
%Add yourself if you have helped and aren't on the list

\Rtodo{The text below needs thorough review by the community.  We are certain that there are many errors, including missing people.  Please send input to fix these errors}

In early 2006, Microsoft issued a call for proposals in the field of computational synthetic biology. A proposal was submitted from UW with the aim to kickstart an effort to develop exchange standards for designs in the new field of synthetic biology. Along with five other groups, the UW group was successful in securing a modest grant. Part of the funds were use to fund the initial standards meeting held in Seattle in April 26-27, 2008. The organizers of the initial meeting were Herbert Sauro, Sean Sleight and Deepak Chandran. The meeting included talks by Raik Gruenberg,  Kim de Mora from the Jason Kelly lab, John Cumbers,  Christopher Anderson, Mac Cowell, Jason Morrison, Jean Peccoud, Ralph Santos, Andrew Milar, Vincent Rouilly, Mike Hucka, Michael Blinov, Lucian Smith, Sarah Richardson, Guillermo Rodrigo, Jonathan Goler, and last but not least Mike Galdzicki. Mike was to go on and lead the development of PoBol, as it was then called. Mike's early efforts were instrumental in making SBOL the success it is today. He organized annual workshops from 2008 to 2011 and kept the idea of developing a standard alive. These were held at the Synthetic Biology Data Exchange Working Group meeting at Stanford in July 26, 2009 and Anaheim, CA on June 13, 2010. Included at the Anaheim meeting were Chandran, Densmore, Dmytriv, Galdzicki, Ham, Rodriquez, Peccoud, Sauro, and Stan. The original SBOL 1.0 was developed at these early meetings through Mike's efforts together with the small group of dedicated researchers. It was also that the Anaheim meeting that a decision was made to write a letter to Nature Biotechnology highlighting the issue of reproducibility in synthetic biology. This letter was initiated by Jean Peccoud and submitted by participants of the Anaheim meeting. 

An important meeting was held in 2011 at Blacksburg, Virginia on January 7-10, 2011 where new members joined the group resulting in 17 individuals at the meeting. New members included Cesar Rodriguez, Mandy Wilson, Jacob Beal, Guy-Bart Stan, Chris Myers, and Nicholas Roehner, and the over all pace of development quickened. 

At a meeting in San Diego in June 2011, the SBOL Developers Group was officially established, rules of governance were established, and the first SBOL editors were elected: Mike Galdzicki, Cesar Rodriguez, and Mandy Wilson.  At this time, Allan Kuchinsky, a research scientist at Agilent, joined the effort, and he was able to obtain some support to begin what was to become libSBOLj.  Kevin Clancy from LifeTechnologies also joined at this time, as well as, Anil Wipat, Matthew Pocock, and Goksel Misirli from Newcastle University.  In October 2011, SBOL 1.0 was officially released.  At our next meeting in Seattle in January 2012, Herbert Sauro was elected the SBOL Chair, and two new editors were added: Matthew Pocock and Ernst Oberortner.  At this meeting, the first data exchange between software tools using SBOL was conducting when a design was passed from Newcastle University's VirtualParts Repository to Boston University's Eugene tool, and finally to University of Utah's iBioSim tool. 

In March 2012, SBOL 1.1 was released, the version that this document replaces. The 8th SBOL workshop was held in November 2012 at Boston University, and the major topic of discussion was the next version of SBOL.  SBOL 1.1 is limited to describing hierarchical DNA sequences.  Several extensions were discussed at this meeting, such as a means to describe genetic regulation what later turned into interactions, and a means to group components what later turned into modules.  In April 2013, at the 9th SBOL workshop at Newcastle University, the framework for SBOL 2.0 was agreed upon.  Nicholas Roehner, Matthew Pocock, and Ernst Oberortner then began work to create a draft proposal for SBOL 2.0.  In January 2014 at the 10th SBOL workshop, this draft was discussed and many refinements were debated and approved.  Another important decision at this meeting was that SBOL should begin investigating joining the COMBINE community of standards (\url{www.co.mbine.org}). 

In the Spring and Summer of 2014, several important events occurred.  In April, several SBOL representatives attended Harmony in Manchester UK to discuss joining the COMBINE community, which was approved by both sides shortly thereafter.  In May, Herbert Sauro, John Gennari, and Chris Myers received a grant from the National Science Foundation to support SBOL (this document and the supporting software are due in no small part to this support).  

In June, a description and our initial, multi-institutional demonstration of the use of SBOL 1.1 was published in Nature Biotechnology \cite{galdzicki2014synthetic}. In July, Nicholas Roehner presented a proposal for the next version of SBOL at the SEED Conference in Los Angeles \cite{roehner2014proposed}.  Finally, in August 2014, the SBOL community attended their first COMBINE workshop as members of the COMBINE community.  At this meeting, many of the final details of SBOL 2.0 were discussed, and the data model presented here is essentially the result of this meeting.

At the Harmony meeting in April 2015 in Wittenberg, Germany, the work on this specification began in earnest.  The key contributors at this meeting and the previous one were: Bryan Bartley (University of Washington), Jacob Beal (BBN Technologies), Kevin Clancy (ThermoFischer), Bryan Der (MIT), John Gennari (University of Washington), Curtis Madsen (Newcastle University), Goksel Misirli (Newcastle University), Chris J. Myers (University of Utah), Tramy Nguyen (University of Utah), Matthew Pocock (Newcastle University and Turing Ate My Hamster LTD), Jackie Quinn (Google), Nicholas Roehner (Boston University), Herbert M. Sauro (University of Washington), Anil Wipat (Newcastle University), and Zhen Zhang (University of Utah).