% -----------------------------------------------------------------------------
\section{A Brief History of SBOL}
% -----------------------------------------------------------------------------
%Add yourself if you have helped and aren't on the list

The SBOL effort was started in 2006 with the goal of developing a data exchange standard for genetic designs. Herbert Sauro (University of Washington) secured a grant from Microsoft in the field of computational synthetic biology, which was used to fund the initial meeting in Seattle on April 26-27, 2008. This workshop was organized by Herbert Sauro, Sean Sleight, and Deepak Chandran, and included talks by Raik Gruenberg, Kim de Mora, John Cumbers,  Christopher Anderson, Mac Cowell, Jason Morrison, Jean Peccoud, Ralph Santos, Andrew Milar, Vincent Rouilly, Mike Hucka, Michael Blinov, Lucian Smith, Sarah Richardson, Guillermo Rodrigo, Jonathan Goler, and Michal Galdzicki. 

% Go over past meetings and mention who funded them, eg Douglas Densmore.

Michal's early efforts were instrumental in making SBOL successful. As part of his doctoral work, he led the development of PoBol (Provisional BioBrick Language), as SBOL was originally known. He organized annual workshops from 2008 to 2011 and kept the idea of developing a genetic design standard alive. The original SBOL 1.0 was developed by a small group of dedicated researchers calling themselves the Synthetic Biology Data Exchange Working Group, meeting at Stanford in 2009 and Anaheim, CA in 2010.  During the Anaheim meeting, the community decided to write a letter to Nature Biotechnology highlighting the issue of reproducibility in synthetic biology~\cite{Peccoud2011}. This letter was initiated by Jean Peccoud and submitted by participants of the Anaheim meeting, including Deepak Chandran, Douglas Densmore, Dmytriv, Michal Galdzicki, Timothy Ham, Cesar Rodriquez, Jean Peccoud, Herbert Sauro, and Guy-Bart Stan. The overall pace of development quickened when several new members joined at the next workshop in Blacksburg, Virginia on January 7-10, 2011. This early work was also supported by an STTR grant from the National Institute of Health (NIH \#1R41LM010745 and \#9R42HG006737, from 2010-13) in collaboration with Clark \& Parsia, LLC (Co-PIs: John Gennari and Evren Sirin). New members included Cesar Rodriguez, Mandy Wilson, Guy-Bart Stan, Chris Myers, and Nicholas Roehner.

The SBOL Developers Group was officially established at a meeting in San Diego in June 2011.  Rules of governance were established, and the first SBOL editors were elected: Mike Galdzicki, Cesar Rodriguez, and Mandy Wilson. At our next meeting in Seattle in January 2012, Herbert Sauro was elected the SBOL chair, and two new editors were added: Matthew Pocock and Ernst Oberortner.  New developers joining at these workshops included several representatives from industry, Kevin Clancy, Jacob Beal, Aaron Adler, and Fusun Yaman Sirin. New members from Newcastle University included Anil Wipat, Matthew Pocock, and Goksel Misirli.

Development of the first software library (libSBOLj) based on the SBOL standard was initiated by Allan Kuchinsky, a research scientist from Agilent, at the 2011 meeting.  By the time of the 2012 meeting, the first data exchange between software tools using SBOL was conducted when a design was passed from Newcastle University's VirtualParts Repository to Boston University's Eugene tool, and finally to University of Utah's iBioSim tool. 

SBOL 1.0 was officially released in October 2011.  In March 2012, SBOL 1.1 was released, the version that this document replaces. SBOL 1.1 did not make any major changes, but provided a number of small adjustments and clarifications, particularly around the annotation of sequences.  Multi-institutional data exchange using SBOL 1.1 was later demonstrated in Nature Biotechnology \cite{galdzicki2014synthetic}. 

While SBOL 1.1 had a number of significant advantages over the GenBank representation of DNA sequences, such as representing hierarchical organization of DNA components, it was still limited in other respects. The major topic of discussion at the 8th SBOL Workshop at Boston University in November 2012 was how to address these shortcomings through extensions.  Several extensions were discussed at this meeting, such as a means to describe genetic regulation, which later became important classes in the current 2.x specification.  

A general framework for SBOL 2.0 emerged at the 9th SBOL workshop at Newcastle University in April 2013.  Subsequently, Nicholas Roehner, Matthew Pocock, and Ernst Oberortner drafted a proposal for SBOL 2.0, and Nicholas presented this proposal at the SEED conference in Los Angeles in July 2014 \cite{roehner2014proposed}.  The proposed 2.0 data model was discussed over the course of the 10th, 11th, and 12th workshops.  
The SBOL 2.0 specification document was drafted at the 13th workshop in Wittenberg, Germany. The SBOL 2.x data model presented was essentially the result of these meetings and ongoing discussions conducted through the SBOL Developers mailing lists, plus minor adjustments and updates approved by the community through subsequence meetings and mailing list discussions.

From 2014 to 2019, development of SBOL 2.x was funded in large part by a grant from the National Science Foundation (DBI-1355909 and DBI-1356041).  The SBOL 2.x specification documents and the supporting software libraries are due in no small part to this support. Any opinions, findings, and conclusions or recommendations expressed in SBOL materials are those of the author(s) and do not necessarily reflect the views of the National Science Foundation.

The Computational Modeling in Biology Network (\href{http://www.co.mbine.org}{COMBINE}) holds regular workshops at which synthetic biologists and systems biologists work toward a common goal of integrating biological knowledge through interoperable and non-overlapping data standards. Several SBOL Developers proposed that SBOL join this larger standards community after attended a COMBINE workshop in April 2014.  The proposal passed and SBOL workshops have been co-located with COMBINE meetings since the 11th workshop at the University of Southern California in August 2014.

In 2019 the SBOL Industrial Consortium was established as a pre-competitive non-profit organization supporting innovation, dissemination, and integration of SBOL standards, tools and practices for practical applications in an industrial environment. The SBOL Industrial Consortium meets regularly to coordinate its activities, and organises an Industrial Advisory Board to give an industrial perspective on SBOL, as well as providing financial support for projects, activities, and infrastructure within the SBOL community.
Member organsiations include Raytheon BBN Technologies, Doulix, Integrated DNA Technologies, Twist Bioscience, Amyris, Inscripta, Teselagen, Shipyard Toolchains, and Zymergen.

Discussions related to SBOL 3 began at the COMBINE meetings and on the mailing list beginning in the summer of 2018.  Over the next year and a half, several SBOL Enhancement Proposals (SEPs) were written and discussed.  During the early months of 2020, these SEPs were voted on and approved by the SBOL community.  The initial version of the SBOL 3 specification was drafted during HARMONY 2020 at the European Bioinformatics Institute (EBI) in Hinxton, United Kingdom in March 2020.

The authors would also like to thank Michael Hucka for developing the LaTeX style file used to develop this document~\citep{hucka2017sbmlpkgspec}.

