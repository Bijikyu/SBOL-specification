% -----------------------------------------------------------------------------
\section{SBOL Specification Vocabulary}
% -----------------------------------------------------------------------------

\subsection{Term Conventions}

This document indicates requirement levels using the controlled vocabulary specified in IETF RFC 2119 and reiterated in BBF RFC 0.
In particular, the key words "MUST", "MUST NOT", "REQUIRED", "SHALL", "SHALL NOT", "SHOULD", "SHOULD NOT", "RECOMMENDED", "MAY", and "OPTIONAL" in this document are to be interpreted as described in RFC 2119.

\begin{itemize}
\item The words "MUST", "REQUIRED", or "SHALL" mean that the item is an absolute requirement.
\item The phrases "MUST NOT" or "SHALL NOT" mean that the item is an absolute prohibition.
\item The word "SHOULD" or the adjective "RECOMMENDED" mean that there may exist valid reasons in particular circumstances to ignore a particular item, but the full implications must be understood and carefully weighed before choosing a different course.
\item The phrases "SHOULD NOT" or "NOT RECOMMENDED" mean that there may exist valid reasons in particular circumstances when the particular behavior is acceptable or even useful, but the full implications should be understood and the case carefully weighed before implementing any behavior described with this label.
\item The word "MAY" or the adjective "OPTIONAL" mean that an item is truly optional.
\end{itemize}

\subsection{SBOL Class Names}

SBOL defines the following ``top-level'' and dependent classes:

\begin{description}

\item \emph{\sbol{Collection}}:
Represents a user-defined container for organizing a group of SBOL objects.

\item \emph{\sbol{ComponentDefinition}}: Describes the structure of designed entities, such as DNA, RNA, and proteins, as well as other entities they interact with, such as small molecules or environmental properties.

\begin{itemize}
\item \emph{\sbol{Component}}:
Represents a specific occurrence or instance of a single entity within the design of a more complex component.
Each \sbol{Component} is associated with a \sbol{ComponentDefinition}, and there may be many different instances at different locations in a design that share the same definition.

\item \emph{\sbol{Location}}:
Specifies the base coordinates and orientation of a genetic feature on a DNA or RNA molecule or a residue or site on another sequential macromolecule such as a protein.

\item \emph{\sbol{SequenceAnnotation}}:
Describes the \sbol{Location} of a notable sub-sequence found within the \sbol{Sequence} linked to a \sbol{ComponentDefinition}, with an optional link to a \sbol{Component}.

\item \emph{\sbol{SequenceConstraint}}:
Describes the relative spatial position and orientation of two \sbol{Component} objects that are contained within the same \sbol{ComponentDefinition}.
\end{itemize}

\item \emph{\sbol{GenericTopLevel}}:
Represents a data container that can contain custom data added by user applications.

\item \emph{\sbol{Model}}:
Links an SBOL representation of biological components and their interactions to quantitative, computational models that may be used to predict the functional behavior of a biological design.

\item \emph{\sbol{ModuleDefinition}}:
Describes a ``system'' design as a collection of biological components and their functional relationships.

\begin{itemize}
\item \emph{\sbol{FunctionalComponent}}:
Represents a specific occurrence or instance of an \sbol{ComponentDefinition} within a \sbol{ModuleDefinition}.
Exactly like a \sbol{Component}, except that it can be associated with information about its context of use in the \sbol{Module}, rather than in the context of a containing \sbol{ComponentDefinition}.

\item \emph{\sbol{Interaction}}:
Describes a functional relationship between biological entities, such as regulatory activation or repression, or a biological processes such as transcription or translation.

\item \emph{\sbol{MapsTo}}:
When a design (\sbol{ComponentDefinition} or \sbol{ModuleDefinition}) includes another design as a substructure, the larger design may need to refer to a \sbol{ComponentInstance} from the sub-design.
In this case, a copy of the referenced \sbol{ComponentInstance} needs to be created in the design and a \sbol{MapsTo} is added to the instance for the sub-design, which associates the original and the copy.

\item \emph{\sbol{Module}}:
Represents a specific occurrence or instance of a sub-system within a larger design.
Each \sbol{Module} is associated with a \sbol{ModuleDefinition}, and there may be many different instances at different locations in a design that share the same definition.

\item \emph{\sbol{Participation}}:
Describes the role that a \sbol{Component} plays in an \sbol{Interaction}.
For example, a transcription factor might participate in an \sbol{Interaction} as a repressor or as an activator.

\end{itemize}

\item \emph{\sbol{Sequence}}:
Represents a contiguous series of monomers in a macromoleculer polymer such as DNA, RNA, or protein. 

\end{description}