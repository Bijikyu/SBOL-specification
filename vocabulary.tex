% -----------------------------------------------------------------------------
\section{SBOL Specification Vocabulary}
% -----------------------------------------------------------------------------

\subsection{Term Conventions}

This document indicates requirement levels using the controlled vocabulary specified in IETF RFC 2119 and reiterated in BBF RFC 0.
In particular, the key words ``MUST'', ``MUST NOT'', ``REQUIRED'', ``SHALL'', ``SHALL NOT'', ``SHOULD'', ``SHOULD NOT'', ``RECOMMENDED'', ``MAY'', and ``OPTIONAL'' in this document are to be interpreted as described in RFC 2119.

\begin{itemize}
\item The words ``MUST'', ``REQUIRED'', or ``SHALL'' mean that the item is an absolute requirement.
\item The phrases ``MUST NOT'' or ``SHALL NOT'' mean that the item is an absolute prohibition.
\item The word ``SHOULD'' or the adjective ``RECOMMENDED'' mean that there might exist valid reasons in particular circumstances to ignore a particular item, but the full implications need to be understood and carefully weighed before choosing a different course.
\item The phrases ``SHOULD NOT'' or ``NOT RECOMMENDED'' mean that there might exist valid reasons in particular circumstances when the particular behavior is acceptable or even useful, but the full implications needs to be understood and the case carefully weighed before implementing any behavior described with this label.
\item The word ``MAY'' or the adjective ``OPTIONAL'' mean that an item is truly optional.
\end{itemize}

\subsection{SBOL Class Names}

SBOL defines the following ``top-level'' and dependent classes:

\begin{description}

\item \emph{\sbol{Collection}}:
Represents a user-defined container for organizing a group of SBOL objects.

\item \emph{\sbol{Component}}: Describes the structure and/or function of designed entities, such as DNA, RNA, and proteins, as well as other entities they interact with, such as small molecules or environmental properties.

\begin{itemize}
\item \emph{\sbol{SubComponent}}:
Pointer class. Incorporates a child \sbol{Component} \textit{by reference} into exactly one parent \sbol{Component}. Represents a specific occurrence or instance of an entity within the design of a more complex entity. Because the same definition might appear in  multiple designs or multiple times in a single design, a single \sbol{Component} can have zero or more parent \sbol{Component}s, and each such parent-child link requires its own, distinct \sbol{SubComponent}.
% Mike Bissell - Clarified the role of this class to match changes below in section six. Introduced the term "pointer class" after consulting briefly with Chris M. The emphasis on pointer classes is on the definitions they point to. WAS: "Represents a specific occurrence or instance of a single entity within the design of a more complex component. Each Component is associated with a ComponentDefinition, and there might be many different instances at different locations in a design that share the same definition." I felt this was necessary because it is too easy for a novice to get confused about Component vs. ComponentDefinition here, where they first encounter Component, even before ComponentDefinition is defined.

\item \emph{\sbol{Location}}:
Specifies the base coordinates and orientation of a genetic feature on a DNA or RNA molecule or a residue or site on another sequential macromolecule such as a protein.

\item \emph{\sbol{SequenceAnnotation}}:
Describes the \sbol{Location} of a notable sub-sequence found within the \sbol{Sequence} of a \sbol{Component}. Can also link to and effectively position a child \sbol{Component}.
% Mike Bissell - Reworded for clarity of the relationship between ComponentDefinition and Sequence. Broke the original paragraph into two sentences for simplicity. Reworded second sentence to emphasize that a Component is a) a child entity, and b) a link (pointer). Old wording was vague on Component's job within SequenceAnnotation, and old wording made it sound as if the Component was the payload itself, rather than just a reference. WAS: "Describes the Location of a notable sub-sequence found within the Sequence linked to a ComponentDefinition, with an option al link to a Component."

\item \emph{\sbol{SequenceConstraint}}:
Describes the relative spatial position and orientation of two \sbol{SubComponent} objects that are contained within the same \sbol{Component}.
\end{itemize}

\item \emph{\sbol{GenericTopLevel}}:
Represents a data container that can contain custom data added by user applications.

\item \emph{\sbol{Model}}:
Links to quantitative or qualitative computational models that might be used to predict the functional behavior of a biological design.
%Goksel - Updated as above: Links an SBOL representation of biological components and their interactions to quantitative, computational models that might be used to predict the functional behavior of a biological design.

\begin{itemize}

\item \emph{\sbol{Interaction}}:
Describes a functional relationship between biological entities, such as regulatory activation or repression, or a biological process such as transcription or translation.

\item \emph{\sbol{MapsTo}}:
When a design \sbol{Component} includes another design as a sub-design, the parent design might need to refer to a \sbol{SubComponent} in the sub-design.
In this case, a \sbol{MapsTo} needs to be added to the instance for the sub-design, and this \sbol{MapsTo} needs to link between the \sbol{SubComponent} in the sub-design and a \sbol{SubComponent} in the parent design.

\item \emph{\sbol{Participation}}:
Describes the role that a \sbol{SubComponent} plays in an \sbol{Interaction}.
For example, a transcription factor might participate in an \sbol{Interaction} as a repressor or as an activator.

\end{itemize}

\item \emph{\sbol{Sequence}}:
Generally represents a contiguous series of monomers in a macromolecular polymer such as DNA, RNA, or protein. A \sbol{Sequence} can also encode the atoms and bonds of a molecule with non-linear structure (see \ref{sec:Sequence}).

\end{description}
