% -----------------------------------------------------------------------------
\section{SBOL Specification Vocabulary}
% -----------------------------------------------------------------------------

This document indicates requirement levels using the controlled vocabulary specified in IETF RFC 2119 and reiterated in BBF RFC 0.
In particular, the key words "MUST", "MUST NOT", "REQUIRED", "SHALL", "SHALL NOT", "SHOULD", "SHOULD NOT", "RECOMMENDED", "MAY", and "OPTIONAL" in this document are to be interpreted as described in RFC 2119.

\begin{itemize}
\item The words "MUST", "REQUIRED", or "SHALL" mean that the item is an absolute requirement.
\item The phrases "MUST NOT" or "SHALL NOT" mean that the item is an absolute prohibition.
\item The word "SHOULD" or the adjective "RECOMMENDED" mean that there may exist valid reasons in particular circumstances to ignore a particular item, but the full implications must be understood and carefully weighed before choosing a different course.
\item The phrases "SHOULD NOT" or "NOT RECOMMENDED" mean that there may exist valid reasons in particular circumstances when the particular behavior is acceptable or even useful, but the full implications should be understood and the case carefully weighed before implementing any behavior described with this label.
\item The word "MAY" or the adjective "OPTIONAL" mean that an item is truly optional.
\end{itemize}

\subsection{SBOL Class Names}

SBOL defines the following classes:

\begin{description}

\item \emph{\sbol{Collection}}:
Represents a user-defined container for organizing a group of SBOL objects.

\item \emph{\sbol{Component}}:
Represents a specific occurrence or instance of a single entity within the design of a more complex component.
Each \sbol{Component} is associated with a \sbol{ComponentDefinition}, and there may be many different instances at different locations in a design that share the same definition.

\item \emph{\sbol{ComponentDefinition}}: Represents designed entities, such as DNA, RNA, and proteins, as well as other entities they interact with, such as small molecules or environmental properties.

\item \emph{\sbol{FunctionalComponent}}:
Represents a specific occurrence or instance of an entity within the design of a \sbol{ModuleDefinition}.
Exactly like a \sbol{Component}, except that it can be associated with information about its context of use in the \sbol{Module}, rather than in the context of a more complex \sbol{Component}.

\item \emph{\sbol{FunctionalComponent}}:
Specifies an occurrence of a component that is intended to have a biological function inside a design, as opposed to a component whose purpose is simply to compose a biological structure. 

\item \emph{\sbol{GenericTopLevel}}:
Represents a data container that can contain custom data added by user applications.

\item \emph{\sbol{Interaction}}:
Describes a functional relationship between biological effectors, such as regulatory activation or repression.  Interactions can also be used to describe processes from the central dogma of biology, such as transcription and translation.

\item \emph{\sbol{Location}}:
Specifies the base coordinates of a genetic feature on a DNA or RNA molecule or a residue or site on another sequential macromolecules such as a protein.

\item \emph{\sbol{MapsTo}}:
Links \sbol{Component} objects (both Components and FunctionalComponents) from different levels of hierarchy in a design, indicating that they referto the same entity (e.g., a promoter that is the target of a repressor in one module is the same promoter that regulates expression of a reporter in another module).

\item \emph{\sbol{Model}}:
Links a SBOL representation of biological components and their interactions to quantitative, computational models that may be used to predict the functional behavior of a biological design.

\item \emph{\sbol{Module}}:
Represents a specific occurrence or instance of a module within a larger design.  For a given module definition, there may be several instances of modules that specify the overall function of a biological design.

\item \emph{\sbol{ModuleDefinition}}:
Desribes an abstract functional module that may be composed of many functional interactions between biological components.

\item \emph{\sbol{Participation}}:
Describes the role that a component plays in a functional interaction.  For example, a transcription factor may participate in an interaction either as a repressor or activator.

\item \emph{\sbol{Sequence}}:
Represents a contiguous sequence of monomers in a macromoleculer polymer such as DNA, RNA, or protein. The
sequence is a fundamental information object for synthetic biology and is needed to reuse components, to replicate synthetic biology work, and to assemble new synthetic biological systems. Therefore, both experimental work and theoretical sequence composition research depend heavily on the sequence associated with component definitions.

\item \emph{\sbol{SequenceAnnotation}}:
Describes the position and strand orientation (in the case of DNA molecules) of a notable sub-sequence found within the Component being
described. Annotations provide the link which describes the 
sequence of a component in terms of other components (i.e.,
subComponents).

\item \emph{\sbol{SequenceConstraint}}:
Describes the relative spatial orientation of Components which are assembled together into a biological structure.

\end{description}