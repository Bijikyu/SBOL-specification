% -----------------------------------------------------------------------------
\section{Best Practices}
\label{sec:bestpractices}
% -----------------------------------------------------------------------------
\subsection{Versioning}
Currently, if a developer wishes to change a SBOL object that has been published to the Web, then as a best practice they should create a copy of the SBOL object that incorporates the change, but has a new URI. This practice, however, does not inherently involve a standardized declaration that the second object is a version of the first. Consequently, the \sbol{persistentIdentity}, and \sbol{version} data fields have been created to provide developers with the means to declare that a set of SBOL objects are versions of each other (by virtue of having the same persistent URI) and label these objects with version Strings.

\todo[inline]{try to target readers unfamiliar with RDF/XML.  -bder}
%TODO[version]{maybe clarify what type of versioning is this object affecting? TN}

\subsection{Using External Terms}
External ontologies and controlled vocabularies are integral part of SBOL. SBOL utilises these resources to access existing biological information where possible. New SBOL specific terms are defined only when necessary. Instead, SBOL provides placeholders that can point to external terms. For example, types of components, such as DNA or protein, are indicated using BioPAX. Similarly, the role of a DNA component is indicated via the SO terms. Although preferred ontologies have been indicated in relevant sections where possible, other resources providing similar terms can also be used. A summary of these external sources can be found at \ref{tbl:preferred_external_resources}.

\todo[inline]{It would be useful to add URLs to the resources - NeilW}


\begin{table}[ht]
  \begin{edtable}{tabular}{p{3cm}p{3cm}p{4cm}}
    \toprule
    \textbf{SBOL Entity} & \textbf{Property} & \textbf{Preferred External Resource}\\
    \midrule
    \textbf{ComponentDefinition}  & type & BioPAX \\
    						   	  & role & SO (if type is \textit{DNA} or \textit{RNA})    \\
    						   	  & role & CHEBI (if type is \textit{small molecule})    \\
    						   	  & role & UniProt (if type is \textit{protein}??)    \\%TODO What is the external resource for proteins, discuss it.  protein - GO instead of UniProt
\\%TODO Should there be an external listing for all Top Level? 
                                  
    \textbf{Interaction}	      & type & SBO      \\
    \textbf{Participation}	      & type & SBO      \\
    \textbf{Model}	      		  & language & EDAM      \\
    				      		  & framework & SBO      \\
    \bottomrule
  \end{edtable}
  \caption{SBOL properties and preferred external resources to choose values from.}
  \label{tbl:preferred_external_resources}
\end{table}
