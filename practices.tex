% -----------------------------------------------------------------------------
\section{Best Practices}
\label{sec:bestpractices}
% -----------------------------------------------------------------------------
\subsection{Use of the Version Property}

Once an SBOL object has been published where others might have accessed it (e.g., to an online repository), it may be the case that others make copies of the object or else come to depend on the particular contents of the object.  Thus, in order to avoid confusion, if a person wants to change the properties of a published object, the best practice is to do so by making a new copy that incorporates the change, with a new URI.

The relationship between the old and new objects (i.e., that the new object was derived from the old object), however, is not visible unless it is explicitly declared.  This is RECOMMENDED to be done using the \sbol{persistentIdentity}, and \sbol{version} properties. The preferred practice for declaring such a relationship is to use the same \sbol{persistentIdentity} for both objects, and label the newer one as being the newer version. Then, when the new object is published, it can be clear to both humans and machines that this object is intended to replace the one that was published previously.  In this way also, when the user of an object always wants the latest version, they can obtain it by referencing the object with the \sbol{persistentIdentity} rather than the \sbol{identity}.

As stated in \ref{sec:version},  it is RECOMMENDED that version numbering should follow the conventions of semantic versions (\url{http://semver.org/}), particularly as implemented by Maven (\url{http://maven.apache.org/}).  This convention represents versions as sequences of numbers and qualifiers separated by the characters {\tt .} and {\tt -} and compared in lexicographical order (for example, 1 < 1.3.1 < 2.0-beta).  For a full explanation, see the linked resources.

%% \subsection{Creation and Modification Dates}

\NVtodo{Annotations: Annotating with created and modified dates, and how to add them. Is this section needed?  -- CJM}

\subsection{Compliant SBOL Objects}
\label{sec:compliant}

Maintaining unique identity URIs for all SBOL objects is a very challenging implementation task.  To reduce the developer's burden, users of SBOL 2.0 are encouraged to follow a few simple rules when constructing the identity and related fields for SBOL objects.  When these rules are followed, we say that the SBOL object is \emph{compliant}.  The rules are as follows:
\begin{enumerate}
\item The \sbol{identity} of an SBOL object should begin with a \emph{URI prefix} that maps to a domain over which the user has control.  Namely, the user can guarantee uniqueness of identities within this domain.
\item In a compliant SBOL object, the \sbol{persistentIdentity} and \sbol{displayId} properties are required.
\item The \sbol{persistentIdentity} of a compliant top level object must end with a delimiter ('/', '\#', or ':') followed by the \sbol{displayId} of the object. 
\item The \sbol{persistentIdentity} of a compliant child object must begin with the \sbol{persistentIdentity} of its parent object and be immediately followed by a delimiter ('/', '\#', or ':') followed by the \sbol{displayId} of the object.
\item When a SBOL object is not given a \sbol{version}, the \sbol{identity} and \sbol{persistentIdentity} must be equal.
\item When a SBOL object is given a \sbol{version}, the \sbol{identity} must be equal to the "\refObj{persistentIdentity}/\refObj{version}".
\item The \sbol{version} of a compliant child object must be equal to the \sbol{version} of it parent object.
\end{enumerate}

For examples, see any example in this specification, as all have been
formulated using compliant URIs.

\subsection{Annotations: Embedded Objects vs. External References}

When annotating an SBOL model with additional information, there are
two general methods that can be used:
\begin{itemize}
\item Embed the information in the SBOL model, either as non-SBOL
  properties or wrapped in \sbol{GenericTopLevel} objects.
\item Store the information separately and annotate the SBOL model
  with URIs that point to it.
\end{itemize}
In theory, either can be used in any case.  In practice, however,
embedding massive amounts of non-SBOL data into SBOL models is likely
to cause problems for people and software tools trying to manage and
exchange such models.  Therefore, it is RECOMMENDED that small amounts of information (e.g., design notes, preferred graphical layout) be embedded in the SBOL model, while large amounts of information (e.g., the contents of the scientific publication from which a model was derived, flow cytometry data characterizing performance) be linked with URIs pointing to external resources.  The boundary between ``small'' and ``large'' is left deliberately vague, recognizing that it will likely depend on the particulars of a given SBOL application.

\subsection{Completeness and Validation}

RDF documents containing serialized SBOL objects may or may not be
entirely self-contained.  A SBOL document is self-contained or ``complete'' if every SBOL object referred to in the document is contained in the document.  It is RECOMMENDED that serializations be complete whenever practical.  In order words, when serializing an SBOL object, serialize all of the other objects that it points to, then serialize all of the additional objects that they point to, etc., until the document is complete.

It is important to note that there is no guarantee that an RDF document
contains valid SBOL. When an RDF document is de-serialized into SBOL
objects, the program doing so SHOULD verify that all of the property
values encoded therein have the right type (e.g., that the object
pointed to by the \sbol{sequences} property of a
\sbol{ComponentDefinition} really is a \sbol{Sequence} object).
For complete files, this can be carried out readily and entirely
locally.  For files that are not complete, an implementation needs to
either have a means of validating those external references (e.g., by
retrieving them from various repositories) or else to mark them as
being unverified and not depend on their correctness.

\subsection{Recommended Ontologies for External Terms}

External ontologies and controlled vocabularies are an integral part of SBOL. SBOL utilizes these resources to access existing biological information where possible. New SBOL specific terms are defined only when necessary. Instead, SBOL provides placeholders that can point to external terms. For example, types of components, such as DNA or protein, are indicated using BioPAX. Similarly, the role of a DNA component is indicated via the SO terms. Although preferred ontologies have been indicated in relevant sections where possible, other resources providing similar terms can also be used. A summary of these external sources can be found at \ref{tbl:preferred_external_resources}.

\begin{table}[ht]
  \begin{edtable}{tabular}{p{3cm}p{1.5cm}p{4.5cm}p{6cm}}
    \toprule
    \textbf{SBOL Entity} & \textbf{Property} & \textbf{Preferred External Resource}
    & \textbf{More Information} \\
    \midrule
    \textbf{ComponentDefinition}  & types & BioPAX & \url{http://www.biopax.org}\\
    						   	  & roles & SO (\textit{DNA} or \textit{RNA}) & \url{http://www.sequenceontology.org}   \\
    						   	  & roles & CHEBI (\textit{small molecule}) & \url{https://www.ebi.ac.uk/chebi/}   \\
%    						   	  & roles & UniProt (if type is \textit{protein}??) \\   
    \textbf{Interaction}	      & types & SBO (occurring entity branch) & 
    \url{http://www.ebi.ac.uk/sbo/main/} \\
    \textbf{Participation}	      & roles & SBO (participant roles branch) &
    \url{http://www.ebi.ac.uk/sbo/main/} \\
    \textbf{Model}	      		  & language & EDAM & \url{http://bioportal.bioontology.org/ontologies/EDAM}     \\
    				      		  & framework & SBO (modeling framework branch) &
    \url{http://www.ebi.ac.uk/sbo/main/} \\
    \bottomrule
  \end{edtable}
  \caption{Preferred external resources from which to draw values for various SBOL properties.}
  \label{tbl:preferred_external_resources}
\end{table}

\NVtodo{Goksel and Neil need to sort out GO vs. UniProt, and possibly just recommend both here.}